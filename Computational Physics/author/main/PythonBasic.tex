%%%%%%%%%%%%%%%%%%%%% chapter.tex %%%%%%%%%%%%%%%%%%%%%%%%%%%%%%%%%
%
% sample chapter
%
% Use this file as a template for your own input.
%
%%%%%%%%%%%%%%%%%%%%%%%% Springer-Verlag %%%%%%%%%%%%%%%%%%%%%%%%%%
%\motto{Use the template \emph{chapter.tex} to style the various elements of your chapter content.}
\chapter{Python 기초}
\label{ch:1} % Always give a unique label


이 책의 목적은 공리로부터 시작하여 양자 역학에 도입하는 것이다. 이 챕터의 목적은 필수적인 수학적 토대를 세우는 데 있다. 앞으로 필요할 모든 수학은 아마도 여러분이 알고 있다고 생각되는 벡터와 행렬에 대한 기본적인 아이디어로부터 출발하여 전개될 것이다. 수학에 있어 조금의 편안함을 제공하고 여기서 전개된 아이디어의 광범위한 적용 가능성을 보여주기 위해 고전 역학과 관련된 많은 예제와 연습 문제가 제공된다. 여러분이 이 챕터에 쏟은 노력은 충분히 가치가 있을 것이다: 여러분이 이 과정을 들어가는 준비에 도움이 될 뿐만 아니라 단편적으로 배웠을 많은 아이디어들을 통합할 것이다. 다른 챕터와 마찬가지로, 이 챕터를 제대로 공부하기 위해선 많은 문제를 풀어야 한다.

\section{Linear Vector Spaces: Basics}

이 절에서는 \emph{선형 벡터 공간(linear vector spaces)}에 대해 도입할 것이다. 여러분은 기초 물리학에서 속도, 힘, 위치, 돌림힘 등의 크기와 방향을 담고 있는 화살표에 친숙할 것이다. 여러분은 어떻게 그것들을 더하고, 그것들에 스칼라를 곱하는지와 이러한 연산에 따른 규칙도 알고 있다. 예를 들어, 여러분은 스칼라곱이 분배 가능하다는 것을 안다: 두 벡터의 합과의 곱은 곱들의 합과 같다. 우리가 원하는 것은 이러한 단순한 경우를 기본적인 특징이나 공리의 집합으로 추상화하고, 같은 형식을 따르는 것들의 어떤 집합이 선형 벡터 공간이라고 말하는 것이다. 일반화하는 과정에서 어떤 성질을 유지할 지 결정할 때 영리함이 필요하다. 너무 많이 유지하면, 다른 예시가 없을 것이다; 너무 적게 유지하면 공리로부터 유도할 흥미로운 결과가 없을 것이다.

다음 목록은 수학자들이 벡터 공간의 필수 조건으로 현명하게 선택한 특징들이다. 여러분이 이것을 읽을 때, 이들을 화살표의 세계와 비교하고 이들이 정말 친숙한 벡터들이 갖는 특징과 같음을 확인하길 바란다. 하지만 모든 벡터에는 크기와 방향이 있어야 한다는 조건이 눈에 띄게 사라져 있다. 이는 우리가 처음 들었을 때 머리에 박힌 첫 번째이자 가장 두드러진 특징이었다. 따라서 여러분은 이 조건을 버리는 것이 아기가 목욕물과 함께 버려진 것과 같다고 생각할 수 있다. 하지만 벡터 공간이라는 제목 아래에서 다양한 아이디어가 통일되고 종합되는 것을 보면서 이러한 선택 뒤에 숨겨진 지혜를 이해할 충분한 시간을 갖게될 것이다. 

\begin{definition}
다음 조건이 존재하는 선형 벡터 공간 $\mathbb{V}$는 벡터라고 부르는 객체 $\ket{1}$, $\ket{2}$, $\cdots$, $\ket{V}$, $\cdots$, $\ket{W}$, $\cdots$ 들의 집합이다.
\begin{enumerate}
	\item $\ket{V} + \ket{W}$와 같이 쓰는 벡터 합을 형성하는 명확한 규칙,
	\item 다음 특징을 가지며 $a\ket{V}$와 같이 쓰는 스칼라 $a$, $b$, $\cdots$의 곱에 대한 명확한 규칙.
\end{enumerate}
\begin{itemize}
	\item 이러한 연산의 결과는 공간의 또 다른 원소이다. 즉, \textit{닫혀있다}: $\ket{V} + \ket{W} \in \mathbb{V}$.
	\item 스칼라 곱은 \textit{분배 가능하다}: $(a + b)\ket{V} = a\ket{V} + b\ket{V}$.
	\item 
\end{itemize}
\end{definition}