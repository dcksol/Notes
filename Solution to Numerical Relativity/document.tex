\documentclass{article}
\usepackage{setspace}

\usepackage{makeidx}         % allows index generation
\usepackage{graphicx}        % standard LaTeX graphics tool
\usepackage{float}
\usepackage[bottom]{footmisc}% places footnotes at page bottom

\usepackage{multirow}

\usepackage{cancel}
\usepackage{physics}
\usepackage{tensor}
\usepackage{amssymb}
\usepackage{newtxtext}
\usepackage[subscriptcorrection,mtpcal,nofontinfo]{mtpro2}

\begin{document}
\title{Solution to Numerical Relativity}
\maketitle
\tableofcontents

\cleardoublepage

\setcounter{section}{1}
\section{The 3+1 decomposition of Einstein's equations}
\subsubsection*{Exercise 2.1}
\subsubsection*{Exercise 2.2}
From
\begin{equation}
	\mathcal{C}_E = D_i E^i - 4 \pi \rho,
\end{equation}
\begin{equation}
	\partial_t E_i = D_i D^j A_j - D^j D_j A_i - 4 \pi j_i,
\end{equation}
\begin{equation}
	\frac{\partial \rho}{\partial t} + D_i j^i = 0,
\end{equation}
we can get
\begin{align}
	\partial_t \mathcal{C}_E &= D_i \partial_t E^i - 4\pi \frac{\partial \rho}{\partial t}\\
	&= D_i (D^i D^j A_j - D^j D_j A^i - 4\pi j^i) + 4\pi D_i j^i\\
	&= D_i D^i D^j A_j - D_i D^j D_j A^i\\
	&= 0.
\end{align}

\subsubsection*{Exercise 2.5}
\begin{align}
	\omega_{[a}\nabla_b w_{c]} &= \frac{1}{6}(\omega_a \nabla_b \omega_c - \omega_a \nabla_c \omega_b + \omega_b \nabla_c \omega_a - \omega_b \nabla_a \omega_c + \omega_c \nabla_a \omega_b - \omega_c \nabla_b \omega_a).
\end{align}
Using $\omega_a = \alpha \nabla_a t$, the first two term in parenthesis will be
\begin{align}
	\omega_a \nabla_b \omega_c - \omega_a \nabla_c \omega_b &= \alpha(\nabla_a t)(\nabla_b \alpha \nabla_c t + \alpha \nabla_b \nabla_c t) - \alpha(\nabla_a t)(\nabla_c \alpha \nabla_b t + \alpha \nabla_c \nabla_b t)\\
	&= \alpha (\nabla_a t)(\nabla_b \alpha)(\nabla_c t) - \alpha (\nabla_a t)(\nabla_c \alpha)(\nabla_b t).
\end{align}
Therefore, 
\begin{align}
	\omega_{[a}\nabla_b w_{c]} = & \alpha (\nabla_a t)(\nabla_b \alpha)(\nabla_c t) - \alpha (\nabla_a t)(\nabla_c \alpha)(\nabla_b t)\\
	&+\alpha (\nabla_b t)(\nabla_c \alpha)(\nabla_a t) - \alpha (\nabla_b t)(\nabla_a \alpha)(\nabla_c t)\\
	&+\alpha (\nabla_c t)(\nabla_a \alpha)(\nabla_b t) - \alpha (\nabla_c t)(\nabla_b \alpha)(\nabla_a t)\\
	&= 0.
\end{align}

\subsubsection*{Exercise 2.6}
To show that $\gamma^a_bv^b$ is purely spatial, we should show that $n_a\gamma^a_bv^b=0$.
\begin{align}
	n_a\gamma^a_bv^b &= n_a(\delta^a_b+n^an_b)v^b\\
	&= (n_b + n_an^an_b)v^b\\
	&= (n_b - n_b)v^b\\
	&= 0.
\end{align}

\subsubsection*{Exercise 2.7}
\begin{align}
	T_{ab} =& \delta^c_a\delta^d_bT_{cd}\\
	=& (N^c_a+\gamma^c_a)(N^d_b+\gamma^d_b)T_{cd}\\
	=& (N^c_aN^d_b + N^c_a\gamma^d_b + \gamma^c_aN^d_b+\gamma^c_a\gamma^d_b)T_{cd}\\
	=& n_an_bn^cn^dT_{cd} - n_an^c\perp T_{cb} -n_bn^d\perp T_{ad} + \perp T_{ab}.
\end{align}


\subsubsection*{Exercise 2.8}
It is trivial that $\nabla_a g_{bc}=0$. Since $\gamma_{bc} = g_{bc}+n_bn_c$, we only need to show that $D_a(n_bn_c)=0$.
\begin{align}
	D_a(n_bn_c) =& \gamma^d_a\gamma^e_b\gamma^f_c\nabla_d(n_en_f)\\
	=& \gamma^d_a (\delta^e_b+n^en_b)(\delta^f_c+n^fn_c)(n_e\nabla_dn_f + n_f\nabla_dn_e)\\
	=& \gamma^d_a (\delta^e_b\delta^f_c + \delta^e_b n^f n_c + \delta^f_cn^en_b+n^en_bn^fn_c)(n_e\nabla_d n_f + n_f \nabla_dn_e)\\
	=& \gamma^d_a(n_b\nabla_d n_c + n_c\nabla_d n_b + n_bn_cn^f\nabla_dn_f+n_cn_fn^f\nabla_dn_b\\
	&+ n_bn_en^e\nabla_dn_c + n_bn_cn^e\nabla_dn_e\\ &+n_bn_cn^en_en^f\nabla_dn_f+n_bn_cn^en_fn^f\nabla_dn_e\\
	=& n_b \nabla_dn_c + n_c\nabla_d n_b + n_b n_c n^f \nabla_dn_f - n_c\nabla_dn_b\\
	&-n_b\nabla_dn_c+n_bn_cn^e\nabla_dn_e-n_bn_cn^f\nabla_dn_f-n_bn_cn^e\nabla_dn_f\\
	=&0.
\end{align}

\subsubsection*{Exercise 2.9}
If $v^a$ is purely spatial, $n_av^a = 0$ and $\gamma^b_av^a=(\delta^b_a+n^bn_a)v^a=v^b+0$. Similarly $\gamma^a_bw_a=w_b$. So
\begin{align}
	D_a(v^bw_b) =& \gamma^c_a\nabla_a(v^b\omega_b)\\
	=& \gamma^c_a(v^b\nabla_c\omega_b + \omega_b \nabla_c v^b)\\
	=& \gamma^c_a(\gamma^b_dv^d\nabla_c\omega_b + \gamma^d_b\omega_d\nabla_cv^b)\\
	=& v^d D_a\omega_d + \omega_d D_av^d\\
	=& v^b D_a \omega_b + \omega_b D_a v^b. 
\end{align}

\subsubsection*{Exercise 2.12}
We need to show that $n^a\nabla_b n_a = 0$.

\subsubsection*{Bonus}
Start from exercise 2.5 and $\omega_a = -n_a$. Contract with $n^a$,
\begin{align}
	n^an_{[a}\nabla_bn_{c]} =& n^an_a\nabla_bn_c - n^an_a\nabla_c n_b\\
	& + \cancelto{0}{n^an_b\nabla_c n_a} - \cancel{n^an_b\nabla_a n_c}\\
	& + \cancel{n^an_c\nabla_a n_b} - \cancelto{0}{n^an_c\nabla_b n_a}\\
	=& \nabla_cn_b - \nabla_bn_c\\
	=&0.
\end{align}
Therefore $\nabla_bn_c = \nabla_cn_b$.



\section{Constructing initial data}

\end{document}