\documentclass{book}
\usepackage[utf8]{inputenc}
\usepackage{kotex}
\usepackage{amsmath}
\usepackage{amsthm}
\usepackage{setspace} % 줄간격 지원
% \usepackage{unicode-math}
% \usepackage{mathptmx}
\usepackage{physics}
\usepackage{newtxtext}       % 
%\usepackage{newtxmath}       % selects Times Roman as basic font
\usepackage[subscriptcorrection,mtpcal,mtphrd,nofontinfo]{mtpro2}
\usepackage[version=4]{mhchem}
\usepackage{physics}
\setstretch{1.6} % 줄간격 설정
\usepackage{indentfirst} % 매 문단 들여쓰기
\usepackage[a4paper, total={6in, 9in}]{geometry}
\usepackage{bm}
\usepackage{enumitem}

\usepackage{scrextend}% not needed with a KOMA-Script class, provides the
% `addmargin' environment

\usepackage[load-headings]{exsheets}

\DeclareInstance{exsheets-heading}{mylist}{default}{
	runin = true ,
	attach = {
		main[l,vc]number[l,vc](-4em,0pt) ; % 3em = indent of question body
		main[r,vc]points[l,vc](\linewidth+\marginparsep,0pt)
	}
}

\SetupExSheets{
	headings = mylist , % use the new headings instance
	headings-format = \normalfont\bfseries ,
	counter-format = \thechapter.se.qu ,
	counter-within = section
}


\usepackage{etoolbox}
% 3em = indent of question body :
\AtBeginEnvironment{question}{\addmargin[4em]{0em}}
\AtEndEnvironment{question}{\endaddmargin}



\theoremstyle{definition}
\newtheorem{example}{Example}[section]

\renewcommand{\qed}{\hfill$\blacksquare$}
%\newcommand{\qedblack}{\hfill$\blacksquare$}

\title{{\bfseries Arfken} \ }
\author{김동찬}
\date{June 2022}

\begin{document}
	
	\chapter*{Preface}
	오픈소스로 공개되어 수정에 기여할 수 있다.
	
	용어를 정의할 때 \textbf{English(한글)}과 같이 쓰였다면 이후 영어 표현으로, 반대의 경우는 한글 표현으로 서술하겠다는 것을 의미한다. 굳이 괄호로 한글 뜻을 적어둔 것은 대충 무슨 의미인지는 알면 좋을 경우이다.
	
	\setcounter{chapter}{16}
	\chapter{ Group Theory}
	
	\section{서론}

오랫동안 \textbf{대칭성}은 물리 체계를 연구하는 데 중요했다. 결정 체계의 기하적 대칭성과 그들의 x-선 회절 스펙트럼의 연결은 회절 패턴을 해석하고 이로부터 결정 안에서 원자들의 위치에 대한 정보를 추출하는 데 있어 중요하다. 분자들의 기하적 대칭성은 복사를 흡수 또는 방출하는 과정에서 어떤 진동 모드가 활성화 될 것인지 결정한다; 주기율표의 대칭성은 에너지 밴드, 전기 전도성, 초전도성을 함축한다. 위치나 방향에 대한 물리 법칙의 불변성은 (즉, 공간의 대칭성) 선형 및 각 운동량에 대한 \textbf{보존 법칙}을 야기한다. 가끔 대칭성의 불변성이 함축하는 것은 처음 짐작한 것보다 더 복잡하고 정교하다; 서로 다른 일정한 속력으로 운동하는 관찰자의 좌표계(\textbf{관성계})에서 측정이 이루어질 때, 전자기학에 의해 예측된 힘의 불변성은 아인슈타인의 특수 상대성 이론의 발견에 중요한 단서가 되었다. 양자역학의 등장으로, 각운동량과 스핀을 고려하는 것은 물리학에 새로운 대칭성의 개념을 도입하게 되었다. 이러한 아이디어는 입자 이론의 현대적 발전을 촉진시켰다.

이러한 대칭성 개념들의 중심은 대칭성 연산의 완전 집합이 수학에서 군(group)이라고 알려진 것을 형성한다는 사실이다. 그림 17.2에 묘사된 것과 같은 대칭성 연산과 같이 군이 \textbf{유한}하거나 \textbf{이산}적일 때 군의 원소들은 유한한 수일 것이다. 하지만 그렇지 않다면, 대칭성 연산은 수적으로 무한하며 연속적인 변수에 의해 기술될 것이다; 그러한 군은 \textbf{연속}이라 한다. 연속군의 예는 축에 대해 회전하는 원형 물체의 가능한 집합이다(여기서 변수는 회전각이다).

\subsection{군의 정의}

군 $G$는 $G$의 원소라 불리는 객체나 연산(예를 들어  회전이나 다른 변환들)의 집합으로 정의된다. 이 원소들은 $*$으로 나타내고 \textbf{곱}으로 불리는 절차를 통해 결합될 수 있다. 다음과 같은 4가지 조건을 만족하는 잘 정의된 \textbf{곱셈}을 형성할 수 있다:
\begin{enumerate}
	\item $a$, $b$가 $G$의 임의의 두 원소일 때, 곱 $a * b$ 또한 $G$의 원소이다; 좀 더 형식적으로, $a*b$는 $G$의 원소의 순서쌍 $(a, b)$와 연관된다. 다른말로 $G$는 그것의 원소의 곱에 대해 \textbf{닫혀있다}고 한다.
	\item 이러한 곱은 결합 법칙을 만족한다: $(a*b)*c=a*(b*c)$.
	\item $G$에 고유 단위 원소 $I$가 존재하여, $G$의 모든 원소 $a$에 대해 $I*a = a * I = a$를 만족한다.
	\item 각각의 $G$의 원소 $a$는 $a^{-1}$이라 나타내는 역을 가지며 $a*a^{-1} = a^{-1}*a = I$이다.
\end{enumerate}
위의 간단한 규칙들은 다음과 같은 직접적인 결과를 갖는다:
\begin{itemize}
	\item 임의의 원소 $a$의 역이 유일하다는 것을 보일 수 있다: 만약 $a^{-1}$과 $\hat{a}^{-1}$이 모두 $a$의 역이라면, $\hat{a}^{-1} = \hat{a}^{-1} * (a * a^{-1}) = (\hat{a}^{-1}*a)*a^{-1} = a^{-1}$.
	\item $a$는 고정되어 있고, $g$는 군의 모든 원소를 범위로 가질 때, 곱셈 $g*a$는 그룹의 모든 원소를 (어떤 순서로) 구성한다. 만약 $g$와 $g'$가 같은 원소를 생성하면, $g*a=g'*a$이다. 오른쪽에 $a^{-1}$을 곱하면, $(g*a)*a^{-1} = (g'*a)*a^{-1}$에서 $g=g'$을 얻는다.
\end{itemize}
다음은 몇 가지 유용한 규약과 추가적인 정의들이다:
\begin{itemize}
	\item 이산 군이 ($I$를 포함해) $n$개의 원소를 가진다면, 이것의 \textbf{order(차수)}는 $n$이다; order $n$인 연속 군은 $n$개의 매개변수로 정의된 원소를 가진다.
	\item $G$의 모든 원소 $a$, $b$에 대해 $ab=ba$일 때, 이 곱은 \textbf{commutative(교환가능)}하다고 하고, 그 군은 \textbf{abelian(아벨 군)}이라 한다.
	\item 군이 수열 $I$, $a$, $a^2(=aa)$, $a^3$, $\cdots$이 군의 모든 원소를 포함하는 원소 $a$를 가질 때, \textbf{cyclic(순환 군)}이라 한다. 만약 군이 cyclic이면, 또한 abelian이어야 한다. 하지만 모든 abelian 군이 cyclic인 것은 아니다.
	\item 두 군 $\{ I, a, b, \cdots \}$, $\{ I', a', b', \cdots \}$은 이들의 모든 원소를 모든 $a$, $b$에 대해 $ab=c \Longleftrightarrow a'b'=c'$와 같이 일대일 대응시킬 수 있을 때, \textbf{isomorphic(동형)}이다. 만약 대응이 다대일이라면 그 그룹은 \textbf{homomorphic(준동형)}이다.
	\item $G$의 부분집합 $G'$가 $G$에서 정의된 곱셈에 닫혀있다면, 이것 또한 군이고, $G$의 \textbf{부분군(subgroup)}이라 한다. $G$의 항등원 $I$는 항상 $G$의 부분군을 형성한다.
\end{itemize}
\subsection{군의 예}
\begin{example}
	$\bm{D}_3$, \textsf{정삼각형의 대칭성}
	
	정삼각형의 대칭 연산은 6개의 원소를 가지는 유한한 군을 형성한다; 우리의 삼각형은 어떤 면을 위로 향하게 해도 상관 없으며, 상단 위치에 어떤 정점이든 올 수 있다. 초기 방향을 대칭 등가로 변환하는 6개의 연산들로 $I$ (방향이 바뀌지 않는 항등 연산), 삼각형을 반시계 방향으로 $1/3$ 회전시키는 연산인 $C_3$, $C_3$를 두 번 연속 행하는 $C_3^2$ 연산, $1/2$ 회전시키는 연산인 $C_2$ (이 경우 삼각형 평면상의 축을 기준으로), $C_2'$와 $C_2''$ (삼각형 평면상에서 또다른 축을 기준으로 $180^\circ$ 회전하는)가 있다. 그림 은 대칭 연산을 나타내는 개략적인 그림이고, 그림 은 이들의 결과를 나타낸 것으로, 삼각형 정점의 숫자가 각각의 연산들의 효과를 보여준다. 이 군에 대한 곱셈 표가 표 에 나타나 있다. 곱 $ab$가 (먼저 연산 $b$를 적용하고, 연산 $a$를 적용한 결과를 설명하는) 군의 원소로 $a$행 $b$열에 나열되어 있다. 이 군은 몇 가지 이름이 있는데, 그중 하나는 $D_3$이다(``$D$''는 \textbf{dihedral(이면체)}로, 주 대칭축에 수직인 평면에 놓인 회전축에 대해 $180^\circ$ 회전을 나타낸다). 곱셈 표 또는 이들끼리의 대칭 연산을 살펴봄으로써, $I$의 역은 $I$이고, $C_3$의 역은 $C_3^2$ (따라서 $C_3^2$의 역은 $C_3$이다), 그리고 각각의 $C_2$는 스스로의 역임을 알 수 있다. 이 군은 abelian이 아니다; $C_3C_2 \neq C_2C_3$ ($C_3C_2 = C_2''$이지만 $C_2C_3=C_2'$이다). \qed
\end{example}

\begin{example}
	\textsf{원형 디스크의 회전}
	
	대칭축에 대한 원형 디스크의 회전은 원소들이 각 $\phi$의 회전으로 구성된 1차 연속 군을 형성한다. 군의 원소 $\mathsf{R}(\phi)$의 개수는 무한하고, $\phi$는 범위 $(0, 2\pi)$ 안의 임의의 각이다. 항등 원소는 명확히 $\mathsf{R}(0)$이다; $\mathsf{R}(\phi)$의 역은 $\mathsf{R}(2\pi - \phi)$이다. 이 군의 곱셈 법칙은 $\mathsf{R}(\phi)\mathsf{R}(\theta) = \mathsf{R}(\theta)\mathsf{R}(\phi)$이므로, $\mathsf{R}(\phi)\mathsf{R}(\theta) = \mathsf{R}(\theta)\mathsf{R}(\phi)$이고, 이 군은 abelian이다. 디스크 위의 한 점 $(x, y)$가 회전 후 어떻게 되는 지 이해하는 것이 유용하다. 회전은 이 책의 다른 곳에서 사용되는 좌표축의 반시계 방향 회전과 같이 $z$축에 대해 양의 $z$에서 아래로 내려다보며 시계 방향으로 각 $\phi$만큼 이루어진다. 이 점의 최종 위치 $(x',y')$는 행렬 방정식
	\begin{equation}
		\mqty(x' \\ y') = \mqty( \cos \phi & \sin \phi \\ -\sin \phi & \cos \phi ) \mqty(x \\ y)
	\end{equation}
	으로 주어진다. \qed
\end{example}

\begin{example}
	\textsf{추상적인 군}
	
	군은 기하학적 연산을 표현할 필요는 없다. 네 개의 양(원소)들의 집합 $I$, $A$, $B$, $C$을 고려하자. 우리가 이것에 대해 아는 것은 오직 임의의 두 원소를 곱했을 때, 결과는 집합 안의 원소라는 것이다. 이 4-원소 집합의 곱셈 표가 표 에 나타나 있다. 이 원소들은 각각이 역을 갖고(자기자신), 항등 원소($I$)가 있고, 집합이 곱셈에 닫혀있기 때문에 군을 형성한다.
	\qed
\end{example}

\begin{example}
	\textsf{Isomorphism과 homomorphism: 군 $C_4$}
	
	뒤집을 수 없는 정사각형의 대칭 연산은 가끔 $C_4$로 부르는 4원군을 형성하는데 이들 원소들은 각각 $I$, $C_4$ ($90^\circ$ 회전), $C_2$ ($180^\circ$ 회전), $C_4'$ ($270^\circ$ 회전)이다. 네 개의 복소수 $1$, $i$, $-1$, $-i$ 또한 군의 연산이 일반적인 곱셈일 때 군을 형성한다. 이 군들은 isomorphic이고, 두 가지 다른 방식으로 대응시킬 수 있다:
	\begin{equation*}
		I \leftrightarrow 1, C_4 \leftrightarrow i, C_2 \leftrightarrow -1, C_4' \leftrightarrow -i \quad\text{또는}\quad I \leftrightarrow 1, C_4 \leftrightarrow -i, C_2 \leftrightarrow -1, C_4' \leftrightarrow i.
	\end{equation*}
	또한 이 군은 $C_4^2 = C_2$, $C_4^3 = C_4'$, 또는 동등하게 $i^2 = -1$, $i^3 = -i$ 이므로 cyclic이다.
	
	군 $C_4$는 오직 $1$과 $-1$만을 가지는 일반적인 곱셈 군과 2-1 대응을 갖는다: $I$와 $C_2 \leftrightarrow 1$, $C_4$와 $C_4'\leftrightarrow -1$. 이는 homomorphism이다. 모든 군이 가지는 더 사소한 homomorphism은 모든 원소가 항등원에 대응될 때 얻을 수 있다. \qed
\end{example}

\section*{Exercises}

\begin{question}
	\textbf{Vierergruppe}(독일어: 4원군))은 Example 17.1.4에서 소개한 군 $C_4$와 다른 군이다. Vierergruppe는 표 17.2와 같은 곱셈 표를 가진다. 이 군이 cyclic인지와 abelian인지 결정하라.
\end{question}

\begin{question}
	\vspace{-2.0em}
	\begin{enumerate}[label=(\alph*), leftmargin=*, parsep=0em, topsep=0pt]
		\item $n$개의 서로 다른 객체들의 순열이 군의 공준을 만족시킴을 보여라.
		\item 세 개의 물체의 순열에 대해 곱셈 표를 채우고 각각의 순열에 일종의 이름을 붙여라. (제안: 순서를 바꾸지 않는 순열을 $I$로 하라.)
		\item 이 순열 군($S_3$로 명명된)이 $D_3$와 isomorphic임을 보이고 대응하는 연산을 확인하라. 이것이 유일한가?
	\end{enumerate}
\end{question}

\begin{question}
	\textbf{재배열 이론:} 구분되는 원소들 $(I, a, b, \cdots, n)$의 군에 대해, 곱 $(aI, a^2, ab,ac, \cdots, an)$의 집합이 새로운 순서로 군의 모든 원소들을 복제함을 보여라.
\end{question}

\begin{question}
	군 $G$는 원소 $h_i$를 가지는 부분군 $H$를 가진다. 기존의 군 $G$의 고정된 원소이지만 $H$의 원소는 \textbf{아닌} $x$를 가정하자. 변환
	\begin{equation}
		xh_ix^{-1}, \quad i = 1, 2, \cdots
	\end{equation}
	은 \textbf{켤레 부분군(conjugate subgroup)} $xHx^{-1}$을 생성한다. 이 켤레 부분군이 네 가지 군의 공준을 각각 만족시킴을 보여서 군임을 보여라.
\end{question}

\begin{question}
	%		\begin{enumerate}[label=(\alph*), leftmargin=*, parsep=0em]
		\vspace{-2.0em}
		\begin{enumerate}[label=(\alph*), leftmargin=*, parsep=0em, topsep=0pt]
			\item 한 특정한 군이 abelian이다. 두 번째 군은 기존의 군에서 각 원소 $g_i$를 $g_i^{-1}$로 치환하여 얻는다. 두 군이 isomorphic임을 보여라.
			
			\textit{Note.} 이는 $ab=c$일 때, $a^{-1}b^{-1}=c^{-1}$임을 보이는 것을 의미한다.
			\item (a)를 이어서, 두 번째 군이 abelian임을 보여라.
		\end{enumerate}
	\end{question}
	
	\begin{question}
		모든 정수값을 가지는 $l$, $m$, $n$에 대해 $\bm{\mathrm{r}}=(la, ma, na)$에 놓여 있는 동일한 원자로 구성된 입방 결정(cubic crystal)을 고려하자.
		\begin{enumerate}[label=(\alph*), leftmargin=*, parsep=0em, topsep=0pt]
			\item 각각의 직각 좌표축이 사중 대칭 축(fourfold symmetry axis)임을 보여라.
			\item 입방 \textbf{점 군(point group)}은 단순 입방 결정을 불변하게 하고, $l=m=n=0$에 있는 원자를 움직이지 않도록 하는 모든 연산(회전, 반사, 역)으로 구성될 것이다. 양의 좌표축과 음의 좌표축을 치환하는 것을 고려함으로써, 이러한 입방 군이 얼마나 많은 원소를 가질 것인지 예측하라.
		\end{enumerate}
	\end{question}
	
	\begin{question}
		그림 17.3과 같이 평면이 규칙적인 육각형들로 덮여있다.
		\begin{enumerate}[label=(\alph*), leftmargin=*, parsep=0em, topsep=0pt]
			\item 평면과 수직하고 세 육각형의 공통 꼭짓점($A$)을 지나는 축에 대한 회전 대칭을 결정하라. 이는 축이 $n$중 대칭을 가질 때, $n$이 무엇인지 (신중한 설명으로) 보이는 것이다.
			\item (a)를 평면과 수직하고 한 육각형의 기하적 중심에 있는 점($B$)을 지나는 축에 대해 반복하라.
			\item 육각형 평면에서 $180^{\circ}$ 회전이 대칭 원소인 모든 다른 종류의 축을 찾아라 (이는 그 축에 대해 평면을 회전하여 뒤집는 것과 같다).
		\end{enumerate}
	\end{question}

	
	\section{군의 표현}
	
	우리가 여기서 배우는 모든 이산 군과 연속 군은 정사각형 행렬로 나타낼 수 있다. 이 방법으로 군의 각 원소를 행렬과 연관시킬 수 있다는 것을 뜻하며, $\mathsf{U}(a)$가 $a$와 연관된 행렬이고, $\mathsf{U}(b)$가 $b$와 연관된 행렬일 때, 행렬 곱 $\mathsf{U}(a)\mathsf{U}(b)$는 $ab$와 연관된 행렬이 될 것이다. 다시 말해서, 행렬은 군과 동일한 곱셈 표를 가진다. 이들은 유니타리(unitary)가 되도록 선택할 수 있기에 이러한 행렬을 $\mathsf{U}$라 부른다. $\mathsf{U}$가 군의 order와 같은 차원을 가질 필요는 없다.
	
	가끔 라벨로 표현을 구별해야 할 필요가 있다. 표현을 명확하게 하기 위해 일반적으로 채택된 이름을 사용할 수 있다; 일반적인 라벨이 필요할 때, $K$ 또는 $K'$을 사용할 것이다. 따라서 $K$라는 표현으로 행렬 $\mathsf{U}^K(a)$로 구성되는 표현 $K$를 나타낼 수 있다.
	
	
	
	
	\section{대칭성과 물리}
	
	\section{이산 군}
	
	\subsection{그 밖의 이산 군}
	
	\section{직접곱}
	
	\section{대칭 군}
	
	\section{연속 군}
	
	\subsection{리 군과 생성자}
	
	\subsection{$\mathsf{SO}(2)$ 군과 $\mathsf{SO}(3)$ 군}
	

	
	
\end{document}
