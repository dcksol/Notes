%!TEX program = xelatex
\documentclass[10pt]{sintefbeamer}

% packages, font, color, and newcommands
% \usepackage{amsfonts, amsmath, oldgerm, lmodern}
\usepackage{fontenc}
\usepackage{kotex}
\usefonttheme{serif}
\usepackage{setspace} % 줄간격 지원
\usepackage{physics}
% \usepackage[mtpcal]{mtpro2}
\usepackage[subscriptcorrection,mtpcal,nofontinfo]{mtpro2}
\usepackage{etoolbox}
\usepackage{mathtools}
\usepackage{tensor}

\makeatletter
\DeclareRobustCommand{\altLbar}{%
	\mathord{\vphantom{L}\mathpalette\@barredI\relax}%
}

\newcommand{\@barredI}[2]{%
	\ooalign{%
		\hidewidth\hidewidth\@barredIbar#1\hidewidth\cr
		$\m@th#1\mathcal{\altL}$\cr
	}%
}

\newcommand{\@barredIbar}[1]{%
	\check@mathfonts
	\ifx#1\displaystyle
	\fontsize{\f@size}{\z@}%
	\def\@barredIbarkern{0.3}%
	\else
	\ifx#1\textstyle
	\fontsize{\f@size}{\z@}
	\def\@barredIbarkern{0.5}%
	\else
	\ifx#1\scriptstyle
	\fontsize{\sf@size}{\z@}
	\def\@barredIbarkern{0.4}%
	\else
	\fontsize{\ssf@size}{\z@}
	\def\@barredIbarkern{0.47}%
	\fi
	\fi
	\fi
	\usefont{OT1}{cmr}{m}{n}%
	\kern-\@barredIbarkern em 
	\raisebox{-.5ex}{\symbol{'26}}%
}
\makeatother

\setmainhangulfont{KoPubWorld Dotum_Pro }[
  Extension = .otf,
  Path = fonts/,
  Scale = 0.95,
  UprightFont = *Medium,
  BoldFont = *Bold,
  % Since Korean font families have no italic fonts, fake slant
  BoldItalicFont = *Bold,
  AutoFakeSlant = 0.2,
  BoldItalicFeatures = { FakeSlant = 0.2 }
]
\setsanshangulfont{KoPubWorld Batang_Pro }[
  Extension = .otf,
  Path = fonts/,
  Scale = 0.95,
  UprightFont = *Medium,
  BoldFont = *Bold,
  % Since Korean font families have no italic fonts, fake slant
  BoldItalicFont = *Bold,
  AutoFakeSlant = 0.2,
  BoldItalicFeatures = { FakeSlant = 0.2 }
]

% meta-data
\title{2. Specific Heat of Solids}
\subtitle{Solid State Basics}
\author{D. C. Kim}
\date{\today}
% \titlebackground{images/background}

% document body
\begin{document}

\maketitle

\setstretch{1.3} % 줄간격 설정

% \begin{frame}

% This template is a secondary creation of \hrefcol{https://www.overleaf.com/latex/templates/sintef-presentation/jhbhdffczpnx}{SINTEF Presentation} template from \hrefcol{mailto:federico.zenith@sintef.no}{Federico Zenith} \vspace{\baselineskip}

% Following is a brief introduction written by \hrefcol{mailto:federico.zenith@sintef.no}{Federico Zenith} about how to use \LaTeX\ and beamer to prepare slides. All rights reserved by him\vspace{\baselineskip}

% This template is released under \hrefcol{https://creativecommons.org/licenses/by-nc/4.0/legalcode}{Creative Commons CC BY 4.0} license

% \end{frame}


\section{Introduction}

\begin{frame}{Why numerical relativity?}
	\begin{itemize}
		\item How to determine the dynamical evolution of a physical system governed by Einstein's equations of general relativity?
		\item Analytic solutions for the evolution of such systems do not exist.
		\item We have to recast Einstein's 4-dimensional field equations into a form that is suitable for numerical integration.
	\end{itemize}
\end{frame}

\begin{frame}{3+1 decompostion}
	\begin{itemize}
		\item The problem of evolving the gravitational field in GR can be posed in terms of a traditional initial value problem or ``Cauchy'' problem.
		\item The evolution of a general relativistic gravitational field is determined by specifying the metric quantities $g_{ab}$ and $\partial_t g_{ab}$ at a given (initial) instant of time $t$.
		\item In particular, we need to specify the metric field components and their first time derivatives everywhere on some 3-dimensional spacelike hypersurface labeled by coordinate $x^0 = t = \mathrm{constant}.$
	\end{itemize}
\end{frame}


\begin{frame}{3+1 decompostion}
	\begin{itemize}
		\item The different points on this surface are distinguished by their spatial coordinates $x^i$.
		\item Now these metric quantities can be integrated forward in time provided we can obtain from the Einstein field equations expressions for $\partial^2_tg_{ab}$ at all points on the hypersurface.
		\item That way we can integrate these expressions to compute $g_{ab}$ and $\partial_t g_{ab}$ on a new spacelike hypersurface at some new time $t+\delta t$, and then, by repeating the process, obtain $g_{ab}$ for all other points $x^0$ and $x^i$ in the (future) spacetime.
	\end{itemize}
\end{frame}


\section{Hamiltonian Formulation}

\begin{frame}{Introduction}
	\begin{itemize}
		\item Lagrangian formulation of a field theory is ``spacetime covariant.''
		\item Hamiltonian formulation of a field theory requires a breakup of spacetime into space and time.
		\item Indeed, the first step in producing a Hamiltonian formulation of a field theory consists of choosing a time function $t$ and a vector field $t^a$ on a spacetime such that the surfaces, $\Sigma_t$, of constant $t$ are spacelike Cauchy surfaces and such that $t^a\grad_at=1$.
		\item The vector field $t^a$ may be interpreted as describing the ``flow of time'' in the spacetime and can be used to identify each $\Sigma_t$ with the initial surface $\Sigma_0$.
	\end{itemize}
\end{frame}

\begin{frame}{Introduction}
	\begin{itemize}
		\item In performing integrals over $\Sigma_t$, it would be natural in most cases to use the volume element
		\begin{equation}
			{}^{(3)}\epsilon_{abc}=\epsilon_{dabc}n^d
		\end{equation}
		, where $n^d$ is the unit normal to $\Sigma_t$.
		\item However, volume elements will be ``time dependent'' in the sense that
		\begin{equation}
			\altLbar_t \epsilon_{abcd} \neq 0 \quad \mathrm{and} \quad \altLbar_t^{(3)}\epsilon_{abc}\neq 0
		\end{equation}
		, which inconvenient.
		\item Therefore, we shall introduce a fixed volume element $e_{abcd}$ on $M$ satisfying $\altLbar_t e_{abcd}=0$.
	\end{itemize}
\end{frame}

\begin{frame}{Introduction}
	\begin{itemize}
		\item The next step in giving a Hamiltonian formulation is to define a configuration space for the field by specifying what tensor field (or fields) $q$ on $\Sigma_t$ physically describes the instantaneous configuration of the field $\psi$.
		\item The space of possible momenta of the field at a given configuration $q$ then is taken to be the ``cotangent space,'' $V_q^{\ast}$, of the configuration space at $q$.
		\item Since the set of possible configurations of the field is infinite-dimensional, we shall not attempt here to give a precise definition of $V_q^\ast$.
	\end{itemize}
\end{frame}

\begin{frame}{Introduction}
	\begin{itemize}
		\item The final and most nontrivial step required for a Hamiltonian formulation of a field theory is the specification of a functional $H[q, \pi]$ on $\Sigma_t$, called the \emph{Hamiltonian}, which is of the form
		\begin{equation}
			H = \int_{\Sigma_t}\mathcal{H},
		\end{equation}
		where the \emph{Hamiltonian density} $\mathcal{H}$ is the local function of $q$, $\pi$ and of their spatial derivatives up to a finite order, such that the pair of equations,
		\begin{equation}
			\dot{q} \equiv \altLbar_t q = \fdv{H}{\pi},
		\end{equation}
		\begin{equation}
			\dot{\pi} \equiv \altLbar_t \pi = - \fdv{H}{q},
		\end{equation}
		is equivalent to the field equation satisfied by $\psi$.
	\end{itemize}
\end{frame}

\section{Einstein Equation in 3+1 Form}

\begin{frame}{The Einstein Equation}
	\begin{itemize}
		\item Consider a spacetime $(\mathcal{M}, \vb*{g})$ such that $\vb*{g}$ obeys the Einstein equation:
		\begin{equation}
			\label{eq5.1}
			\tensor[^4]{\vb*{R}}{}  - \dfrac{1}{2}{}^4 R\vb*{g} = 8 \pi \vb*{T},
		\end{equation}
		where $\tensor[^4]{\vb*{R}}{}$ is the Ricci tensor associated with $\vb*{g}$, $\tensor[^4]{R}{}$ the corresponding Ricci scalar, and $\vb*{T}$ the matter stress-energy tensor.
		\item We are using units in which Newton's gravitational constant $G$ is set to unity, otherwise, the coefficient in front of $\vb*{T}$ in Eq.(\ref{eq5.1}) should read $8\pi G$, and even $8\pi G / c^4$ if we are relaxing $c=1$.
	\end{itemize}
\end{frame}

\begin{frame}{The Einstein Equation}
	\begin{itemize}
		\item We shall also use the equivalent form
		\begin{equation}
			\label{eq5.2}
			\tensor[^4]{\vb*{R}}{} = 8\pi \qty(\vb*{T} - \dfrac{1}{2}T\vb*{g}),
		\end{equation}
		where $T:=g^{\mu\nu}T_{\mu\nu}$ stands for the trace (with respect to $\vb*{g}$) of $\vb*{T}$.
		\item Let us assume that the spacetime $(\mathcal{M}, \vb*{g})$ is globally hyperbolic and let $(\Sigma_t)_{t\in \mathbb{R}}$ be a foliation of $\mathcal{M}$ by a family of spacelike hypersurfaces.
		\item The 3+1 formalism for GR amounts to projecting the Einstein equation onto $\Sigma_t$ and perpendicularly to $\Sigma_t$.
		\item To this aim let us first consider the 3+1 decomposition of the stress-energy tensor.
	\end{itemize}
\end{frame}

\begin{frame}{3+1 Decomposition of the Stress-Energy Tensor}
	\begin{itemize}
		\item From the very definition of a stress-energy tensor, the \emph{matter energy density} as measured by the Eulerian observer introduced is
		\begin{equation}
			E:= \vb*{T}(\vb*{n}, \vb*{n}).
		\end{equation}
		\item This follows from the fact that the 4-velocity of the Eulerian observer is the unit normal vector $\vb*{n}$ of the hypersurfaces $\Sigma_t$.
	\end{itemize}
\end{frame}


% \backmatter

\end{document}
