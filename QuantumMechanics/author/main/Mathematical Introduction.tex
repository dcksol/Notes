%%%%%%%%%%%%%%%%%%%%% chapter.tex %%%%%%%%%%%%%%%%%%%%%%%%%%%%%%%%%
%
% sample chapter
%
% Use this file as a template for your own input.
%
%%%%%%%%%%%%%%%%%%%%%%%% Springer-Verlag %%%%%%%%%%%%%%%%%%%%%%%%%%
%\motto{Use the template \emph{chapter.tex} to style the various elements of your chapter content.}
\chapter{Mathematical Introduction}
\label{ch:1} % Always give a unique label
% use \chaptermark{}
% to alter or adjust the chapter heading in the running head

%\abstract*{Each chapter should be preceded by an abstract (no more than 200 words) that summarizes the content. The abstract will appear \textit{online} at \url{www.SpringerLink.com} and be available with unrestricted access. This allows unregistered users to read the abstract as a teaser for the complete chapter.
%Please use the 'starred' version of the new \texttt{abstract} command for typesetting the text of the online abstracts (cf. source file of this chapter template \texttt{abstract}) and include them with the source files of your manuscript. Use the plain \texttt{abstract} command if the abstract is also to appear in the printed version of the book.}

\abstract{이 책의 목적은 공리로부터 양자 역학을 소개하는 것이다. 이 챕터의 목적은 필수적인 수학적 토대를 세우는 데 있다. 앞으로 필요할 모든 수학은 아마도 여러분이 알고 있다고 생각되는 벡터와 행렬에 대한 기본적인 아이디어로부터 출발하여 전개될 것이다. 수학에 있어 조금의 편안함을 제공하고 여기서 전개된 아이디어의 광범위한 적용 가능성을 보여주기 위해 많은 예제와 연습 문제가 제공된다. 여러분이 이 챕터에 쏟은 노력은 충분히 가치가 있을 것이다: 여러분이 이 과정을 들어가는 준비에 도움이 될 뿐만 아니라 단편적으로 배웠을 많은 아이디어들을 통합할 것이다. 다른 챕터와 마찬가지로, 이 챕터를 제대로 공부하기 위해선 많은 문제를 풀어야 한다.}

\section{Linear Vector Spaces: Basics}

이 섹션은 \emph{선형 벡터 공간(linear vector spaces)}에 대해 소개할 것이다. 

\section{Disjoint Set}
\label{sec:disjointset}
경로 압축을 통한 \texttt{find} 연산의 최적화와 랭크에 의한 합치기를 통한 \texttt{union} 연산의 최적화로 상호 배타적으로 이루어진 집합을 합치거나 어떤 원소의 집합을 판별하는 작업을 거의 $\mathcal{O}(1)$에 해낼 수 있다.

