%%%%%%%%%%%%%%%%%%%%% chapter.tex %%%%%%%%%%%%%%%%%%%%%%%%%%%%%%%%%
%
% sample chapter
%
% Use this file as a template for your own input.
%
%%%%%%%%%%%%%%%%%%%%%%%% Springer-Verlag %%%%%%%%%%%%%%%%%%%%%%%%%%
%\motto{Use the template \emph{chapter.tex} to style the various elements of your chapter content.}
\chapter{What the ancients knew}
어떻게 organisms가 높은 ordered로 살 수 있는가? 에너지의 흐름이 order를 증가시킨다.

\section{Heat}

\subsection{Heat is a form of energy}

\begin{quotation}
마찰은 mechanical energy를 thermal form으로 변환한다. Thermal energy가 적절히 구성되면 energy의 균형이 유지된다.
\end{quotation}

sunlight, potential energy of a rock은 높은 quality, thermal energy는 poor quality를 가진다. Quality of energy는 항상 어떤 과정을 거쳐 degrade된다.

\subsection{Just a little history}

마찰에 의해 생기는 열은 마찰에 반하는 역학적 일에 어떤 상수를 곱한 값과 같다.
\begin{equation}
	(\text{heat produced}) = (\text{mechanical energy input}) \times (0.24\,\mathrm{cal/J})
\end{equation}

\begin{quotation}
System이 original state로 가는 process에 있다고 가정하자(cyclic process이다). system에 가해지는 전체 역학적 일은, 그리고 system에 의한 일은 전체 열과 같다.
\end{quotation}

\subsection{Preview: The concept of free energy}
자유 에너지는 다음과 같은 간단한 식으로 주어진다.
\begin{equation}
	F = E - TS
\end{equation}

\begin{quotation}
	System이 고정된 온도 $T$에 있을 때, 어떤 과정의 전체 효과가 system의 free energy $F$를 감소시킨다면 이 과정은 자발적으로 진행된다. 따라서 만약 system의 free energy가 이미 최소라면, 어떠한 자발적 변화도 일어나지 않을 것이다.
\end{quotation}

\section{How life generates order}
\subsection{The puzzle of biological order}
제2 법칙은 오직 \emph{isolated} system에서만 적용된다.

\begin{quotation}
	Energy의 flow는 order를 증가시킬 수 있다.
\end{quotation}

\subsection{A paradigm for free energy transduction}

\runinhead{Osmotic flow} 처음에는 sugar가 오른편에서 uncovered하지만, dissolve되고 spread되어 오른 chamber로 가면, mysterious한 force가 피스톤을 오른쪽으로 밀기 시작한다. 이러한 과정은 \textbf{osmotic flow}라 한다.

system은 주위에서 \emph{heat}를 흡수한다. 이러한 thermal energy는 mechanical work로 변환된다. order가 paying되어 이러한 일이 일어날 수 있다. Osmotic flow는 \emph{molecular order}을 희생하여 random thermal notion을 load에 반하는 mechanical motion으로 organize한다.

\textbf{reverse osmosis}(or ``ultrafiltration'')은 \emph{system에 energy를 흘려줌으로써 mechanical form을 thermal form으로 degrade하고 order를 증가시킨다.}

\runinhead{Preview: Disorder as information} Room temperature에서
\begin{equation}
	\label{eq:1.7}
	(\text{maximum work}) \approx N \times (4.1\times 10^{-21}\,\mathrm{J}\times \gamma).
\end{equation}
여기서 $N$은 dissolved sugar molecules의 개수이다. $\gamma$는 지금은 중요하지 않은 수치적 상수이다. 식 (\ref{eq:1.7})은 entropy의 의미에 대해 말해준다.
\begin{equation}
	T\Delta S \approx N \times (4.1 \times 10^{-21}\,\mathrm{J}\times\gamma).
\end{equation}

처음 sugar 분자는 전체 volume의 절반에 confined되어 있었다. 그리고 마지막에 그들은 confined되지 않았다. Pistons의 움직임과 같은 손실은 chamber의 절반 중 어디에 sugar molecule이 들어있냐는 것과 같은 정보이다.
\begin{equation}
	\Delta S = \text{constant} \times (\text{number of bits lost}).
\end{equation}

\section{Excursion: Commercials, philosophy, pragmatics}

\section{How to do better on exams (and discover new physical laws)}

\section{Other key ideas from physics and chemistry}
\subsection{Molecules are small}

\subsection{Molecules are particular spatial arrangements of atoms}

\subsection{Molecules have definite internal energies}

\subsection{Low-density gases obey a universal law}

\begin{equation}
	k_\mathrm{B} T_\mathrm{r} \approx 4.1\, \mathrm{pN}\cdot \mathrm{nm}.
\end{equation}




